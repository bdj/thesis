\chapter{The Racket Module System}
Modules are the basic building block of programs written in Racket.
A Racket module specifies the language it is written in, definitions, expressions (run for their effect), and imports and exports. 
Modules usually correspond to files, but multiple modules can be defined in a single file through the use of submodules.
A module's language controls all aspects of what the module means, from how it is parsed to what it means to do function application. 
Usually the language definition is just another Racket module. 
Racket definitions can be normal runtime definitions or compile-time definitions.
The Racket compiler separates compile-time and runtime definitions into different phases, with compile-time definitions at phase 1 and runtime definitions at phase 0.
Macros are special definitions with phase 0 bindings, but with phase 1 values. 
Macros allow programmers to add new features to their programs that can't be added through normal function definitions. 
Racket also allows for definitions directly at phase 1 so that programmers can write helper code for macros through \racket{begin-for-syntax}.

Imports in Racket use the \racket{require} form, which allows for importing definitions from other modules in a variety of ways. 
\racket{require} lets the programmer specify which identifiers to import and any renamings to use.
Also, the programmer can specify at what phase to import another module so they can use other modules to help write macros.
Exports in Racket use the \racket{provide} form, which has similar features as \racket{require}, such as control over which identifiers are exported, renaming capabilities, and phase shifting capabilities.

The following example illustrates some of the features of the racket module system. 

\emph{TODO: racket module example}

The \texttt{obj.rkt} file contains the \texttt{obj} module.
The \texttt{obj} module contains a variable definition, and an accessor and mutator function which are exported.
The \texttt{while-test.rkt} file contains a module written in the \texttt{while-lang} language, which adds a \racket{while} loop form to Racket (Racket has many sophisticated ways to write loops, but none of them use the keyword \racket{while}).
The \texttt{while-lang} language is just another Racket module, contained in \texttt{while-lang.rkt}.
The \texttt{while-lang} module defines a \racket{while} loop as a macro that transforms into a combination of a \racket{loop} and an \racket{if}. 
Also, the module defines a special \mintinline{racket}{#%module-begin} macro which is a hook that will run for every module written in the \texttt{while-lang} language. 
Next, the module includes a \racket{begin-for-syntax} expression. 
All of the code written inside a \racket{begin-for-syntax} expression is shifted to phase 1 (compile-time).
Therefore, the \racket{(update-val)} and \racket{(printf)} expressions will run when the module is compiled. 

