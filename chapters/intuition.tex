\chapter{Intuition}
\label{chap:intuition}
Demodularization is the process of collecting all phase 0 code required by a program into a single module.
This is done by tracing through the \racket{require} graph starting at a program's main module.
The following example program illustrates the need for demodularization and how it is done.

\section{Example Program}
A programmer wants to use a queue library where the library uses different backing structures depending on the length of the queue. 
Listing~\ref{lst:main.rkt} shows an example of using such a library.
\begin{listing}
  \inputminted{racket}{listings/main.rkt}
  \caption{\texttt{main.rkt} module with queue usage}
  \label{lst:main.rkt}
\end{listing}
The library is implemented as a macro, shown in Listing~\ref{lst:queue.rkt}, that switches between two implementations depending on the length of the initial queue.

\begin{listing}
  \inputminted{racket}{listings/queue.rkt}
  \caption{\texttt{queue.rkt} module}
  \label{lst:queue.rkt}
\end{listing}

The macro \racket{with-queue} uses \racket{syntax-case} for its pattern-matching capabilities, and creates a syntax object that will use the long or short implementation depending on the length of \racket{(v ...)}. 
The macro uses a feature from the \texttt{racket/syntax} module called \racket{format-id} which operates like \racket{printf} but instead of creating a string and printing it, it creates a syntax object for an identifier. 
So, \racket{make-queue} is a syntax object that will refer to either \racket{make-long-queue} or \racket{make-short-queue}.
Similarly, \racket{enqueue} and \racket{dequeue} will refer to syntax objects for the correct kind of queue.
The final syntax object is created using \mintinline{racket}{#`} which creates a syntax object but allows for escaping with \mintinline{racket}{#,} to insert the actual syntax objects created above.
All of this work with syntax objects makes it so that the correct queue will be used at compile-time and there will be no overhead of choosing a queue implementation at runtime, and it will all be invisible to the user of the \racket{with-queue} macro.

Listing~\ref{lst:long-queue.rkt} and Listing~\ref{lst:short-queue.rkt} show the two queue implementations, with the \texttt{long-queue} using vectors and the \texttt{short-queue} using lists. 
The actual implementations are not important for this example, so most of the code is elided.
The \racket{...} notation has meaning in Racket, so \racket{---} is used to indicate elided code.

\begin{listing}
  \inputminted{racket}{listings/long-queue.rkt}
  \caption{\texttt{long-queue.rkt} module}
  \label{lst:long-queue.rkt}
\end{listing}

\begin{listing}
  \inputminted{racket}{listings/short-queue.rkt}
  \caption{\texttt{short-queue.rkt} module}
  \label{lst:short-queue.rkt}
\end{listing}

The require graph for this program is shown in Figure~\ref{fig:queue-modules.tex}.
The \texttt{main} module only includes the \texttt{queue} module, and the \texttt{queue} module includes two modules using \racket{for-syntax} and two using \racket{require}.

\begin{figure}
  \begin{tikzpicture}
  [node distance=.5in and 1in,
   module/.style={rectangle,draw,inner sep=.125in}
  ]
\node (main) [module] {main};
\node (queue) [module] [right=of main] {queue}
  edge [<-, thick] node[auto]{\racket{require}} (main);
\node (short) [module] [right=of queue] {short-queue}
  edge [<-, thick] node[auto]{\racket{require}} (queue);
\node (long) [module] [below=of short] {long-queue}
  edge [<-, thick] node[auto]{\racket{require}} (queue);
\node (racket-syntax) [module] [above=of short] {racket/syntax}
  edge [<-, thick] node[auto]{\racket{for-syntax}} (queue);
\node (racket-base) [module] [above=of racket-syntax] {racket/base}
  edge [<-, thick] node[left]{\racket{for-syntax}} (queue);

\end{tikzpicture}

  \caption{Require graph for the \racket{main} program}
  \label{fig:queue-modules.tex}
\end{figure}

Compiling this program involves expanding the \racket{with-queue} macro, which will turn the program into Listing~\ref{lst:main-expanded.rkt}.

\begin{listing}
  \inputminted{racket}{listings/main-expanded.rkt}
  \caption{\texttt{main.rkt} module after macro expansion}
  \label{lst:main-expanded.rkt}
\end{listing}

The expanded program contains references to \racket{long-queue} operations because the length of the initial queue is over 5.
Although the expanded program is not fully compiled to bytecode, it can stand in for a fully compiled program with regard to optimization and demodularization.

With the expanded program, it is possible to see how cross-module optimizations would be difficult.
The program has calls to functions in the \racket{long-queue} module, so the optimizer does not have access to the definitions of the functions while finishing the compilation of the main module.

\section{Demodularization}

Demodularization of the program proceeds by following the require graph and including all phase 0 code from the required modules into the main module of the program. 
The \racket{main} module requires the \racket{queue} module, but the \racket{queue} module only has a macro definition, which is phase 1 code, so it is not put into the main module.
The \racket{queue} module requires \texttt{racket/base} and \texttt{racket/syntax} using \racket{for-syntax}, which means that all of their definitions are imported at phase 1 and do not need to be included in the output (it is possible that they have definitions for phase -1 which would need to be included, but in this simple example we will ignore that). 
Next, demodularization reaches the imports for the \racket{short-queue} and \racket{long-queue} modules and includes all of their definitions (which are all phase 0) in the order they appear in their original modules.
The demodularization algorithm then removes all imports from the main module of the program because all code has been included directly in the main module.
Listing~\ref{lst:main-demodularized.rkt} shows what the demodularized example program would look like.

\begin{listing}
  \inputminted{racket}{listings/main-demodularized.rkt}
  \caption{\texttt{main.rkt} module after demodularization}
  \label{lst:main-demodularized.rkt}
\end{listing}

% DONE: make a point that extra queue implementation is there
The demodularized program should have the exact same behavior at runtime as the modular program.
At this point, the only thing that changes is that the Racket runtime does not need to follow the \racket{require} graph because all phase 0 definitions and expressions are in a single module.
Also, both queue implementations appear in the demodularized program even though only one of them is used.


\section{Optimization}

After a program has been demodularized, the optimizer now has all the information it needs to do optimizations on the whole program.
Listing~\ref{lst:main-optimized.rkt} shows what the example program would look like after optimization.

\begin{listing}
  \inputminted{racket}{listings/main-optimized.rkt}
  \caption{\texttt{main.rkt} module after optimization}
  \label{lst:main-optimized.rkt}
\end{listing}

% DONE: point out that the dead code is eliminated
The existing optimizer is able to inline the various functions because it has access to all of the definitions within the module.
The optimizer also removes the extra queue implementation because it is dead code.
Therefore, demodularization enables optimization by providing the optimizer access to the whole program in one place.
