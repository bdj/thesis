\chapter{Intuition}
\label{chap:intuition}
Demodularization is the process of collecting all phase 0 code required by a program into a single module.
This is done by tracing through the \racket{require} graph starting at a program's main module.
The following example program illustrates the need for demodularization and how it is done.

\section{Example Program}
A programmer wants to use a queue library where the library uses different backing structures depending on the length of the queue. 
Listing~\ref{lst:main.rkt} shows an example of using such a library.
\begin{listing}
  \inputminted{racket}{listings/main.rkt}
  \caption{\texttt{main.rkt} module with queue usage}
  \label{lst:main.rkt}
\end{listing}
The library is implemented as a macro, shown in Listing~\ref{lst:queue.rkt}, that switches between two implementations depending on the length of the initial queue.

\begin{listing}
  \inputminted{racket}{listings/queue.rkt}
  \caption{\texttt{queue.rkt} module}
  \label{lst:queue.rkt}
\end{listing}

The macro \racket{with-queue} uses \racket{syntax-case} for its pattern-matching capabilities, and creates a syntax object that will use the long or short implementation depending on the length of \racket{(v ...)}. 
The macro uses a feature from the \texttt{racket/syntax} module called \racket{format-id} which operates like \racket{printf} but instead of creating a string and printing it, it creates a syntax object for an identifier. 
So, \racket{make-queue} is a syntax object that will refer to either \racket{make-long-queue} or \racket{make-short-queue}.
Similarly, \racket{enqueue} and \racket{dequeue} will refer to syntax objects for the correct kind of queue.

The final syntax object is created using \mintinline{racket}{#`} which creates a syntax object but allows for escaping with \mintinline{racket}{#,}. 
Escaping means that the value of escaped form will be used rather than the literal form. 
In this example, that means that \mintinline{racket}{#,make-queue} will be \racket{make-long-queue} in the output of the macro rather than \racket{make-queue}.
All of this work with syntax objects makes it so that the correct queue will be chosen at compile-time.
There will be no overhead of choosing a queue implementation at runtime, and the whole process will be invisible to the user of the \racket{with-queue} macro.

Listing~\ref{lst:queues.rkt} show the public interfaces for two queue modules, with the \texttt{long-queue} using vectors and the \texttt{short-queue} using lists. 
The actual implementations are not important for this example, so most of the code is elided.
The \racket{...} notation has meaning in Racket, so \racket{---} is used to indicate elided code.


\newsavebox{\longqueue}
\begin{lrbox}{\longqueue}
  \begin{minipage}{.5\textwidth}
    \centering
  \inputminted{racket}{listings/long-queue.rkt} 
\end{minipage}
\end{lrbox}

\newsavebox{\shortqueue}
\begin{lrbox}{\shortqueue}
  \begin{minipage}{.5\textwidth}
    \centering
  \inputminted{racket}{listings/short-queue.rkt} 
\end{minipage}
\end{lrbox}

\begin{listing}
  \begin{tabular}{c c}
    \usebox{\longqueue}
    &
    \usebox{\shortqueue}
    \\
    \rule{0pt}{1em} \\
    (a) & (b)
  \end{tabular}
  \caption{\texttt{long-queue.rkt} (a) and \texttt{short-queue.rkt} (b) modules}
  \label{lst:queues.rkt}
\end{listing}


The require graph for this program is shown in Figure~\ref{fig:queue-modules.tex}.
The \texttt{main} module only includes the \texttt{queue} module, and the \texttt{queue} module includes two modules using \racket{for-syntax} and two using \racket{require}.

\begin{figure}
  \begin{tikzpicture}
  [node distance=.5in and 1in,
   module/.style={rectangle,draw,inner sep=.125in}
  ]
\node (main) [module] {main};
\node (queue) [module] [right=of main] {queue}
  edge [<-, thick] node[auto]{\racket{require}} (main);
\node (short) [module] [right=of queue] {short-queue}
  edge [<-, thick] node[auto]{\racket{require}} (queue);
\node (long) [module] [below=of short] {long-queue}
  edge [<-, thick] node[auto]{\racket{require}} (queue);
\node (racket-syntax) [module] [above=of short] {racket/syntax}
  edge [<-, thick] node[auto]{\racket{for-syntax}} (queue);
\node (racket-base) [module] [above=of racket-syntax] {racket/base}
  edge [<-, thick] node[left]{\racket{for-syntax}} (queue);

\end{tikzpicture}

  \caption{Require graph for the \racket{main} program}
  \label{fig:queue-modules.tex}
\end{figure}

Compiling this program involves expanding the \racket{with-queue} macro in a manner similar to that explained in Chapter~\ref{chap:module-system}. The final phase 0 program before it is turned into bytecode is shown in Listing~\ref{lst:main-expanded.rkt}.
The expanded program contains references to \racket{long-queue} operations because the length of the initial queue is over 5.
This form is sufficient to illustrate the difficulty of optimizing separately compiled modules.
The program has calls to functions in the \racket{long-queue} module, but the optimizer does not have access to the functions.
The optimizer must treat the functions as if they are black boxes, so it cannot inline them.

\begin{listing}
  \inputminted{racket}{listings/main-expanded.rkt}
  \caption{\texttt{main.rkt} module after macro expansion}
  \label{lst:main-expanded.rkt}
\end{listing}


\section{Demodularization}

Demodularization of the program proceeds by following the require graph and including all phase 0 code from the required modules into the main module of the program. 
At this point all of the modules have been separately compiled, so all macro expansion has been done already.
The require graph (Figure~\ref{queue-modules.tex}), starting at the \racket{main} module, requires the \racket{queue} module, but the \racket{queue} module only has a macro definition, which is phase 1 code, so it is not put into the main module.
Again following the graph, the \racket{queue} module requires \texttt{racket/base} and \texttt{racket/syntax} using \racket{for-syntax}, which means that all of their definitions are imported at phase 1 and do not need to be included in the output (it is possible that they have definitions for phase -1 which would need to be included, but in this simple example we will ignore that). 
Next, demodularization reaches the edges in the graph corresponding to the imports for the \racket{short-queue} and \racket{long-queue} modules and includes all of their definitions (which are all phase 0) in the order they appear in their original modules.
The demodularization algorithm then removes all imports from the main module of the program because all code has been included directly in the main module.
Listing~\ref{lst:main-demodularized.rkt} shows what the demodularized example program would look like.

\begin{listing}
  \inputminted{racket}{listings/main-demodularized.rkt}
  \caption{\texttt{main.rkt} module after demodularization}
  \label{lst:main-demodularized.rkt}
\end{listing}

% DONE: make a point that extra queue implementation is there
The demodularized program should have the exact same behavior at runtime as the modular program.
The only thing that changes is that require graph of the program has a single node in it with no edges, and all phase 0 code is in the main module. 
Also, both queue implementations appear in the demodularized program even though only one of them is used.
This extra code is addressed in the optimization pass.

\section{Optimization}

More optimizations can be applied to a demodularized program because such a program has no dependencies on external code.
The Racket optimizer is able to inline the queue operations and remove the dead code associated with the \texttt{short-queue} module. 
The final version of the example program after optimization is shown in Listing~\ref{lst:main-optimized.rkt}. 
The code has a smaller footprint and avoids function call overhead when interacting with the queue.
This example shows that many optimizations are enabled through demodularization of a separately compiled Racket program.

\begin{listing}
  \inputminted{racket}{listings/main-optimized.rkt}
  \caption{\texttt{main.rkt} module after optimization}
  \label{lst:main-optimized.rkt}
\end{listing}

