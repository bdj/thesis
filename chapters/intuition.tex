\chapter{Intuition}
Demodularization is the process of collecting all phase 0 code required by a program into a single module.
This is done by tracing through the \racket{require} graph starting at a program's main module.
The following example program illustrates the need for demodularization and how it is done.

\section{Example Program}
A programmer wants to use a queue library where the library uses different backing structures depending on the length of the queue. 
Listing~\ref{lst:main.rkt} shows an example of using such a library.
\begin{listing}
  \inputminted{racket}{listings/main.rkt}
  \caption{\texttt{main.rkt} module with queue usage}
  \label{lst:main.rkt}
\end{listing}
The library is implemented as a macro, show in Listing~\ref{lst:queue.rkt}, that switches between two implementatations depending on the length of the initial queue.
\begin{listing}
  \inputminted{racket}{listings/queue.rkt}
  \caption{\texttt{queue.rkt} module}
  \label{lst:queue.rkt}
\end{listing}
Listing~\ref{lst:long-queue.rkt} and Listing~\ref{lst:short-queue.rkt} show the two queue implementations, with the \texttt{long-queue} using vectors and the \texttt{short-queue} using lists. 
% TODO: say that ... is actual racket syntax
The \racket{....} signifies elided code.

\begin{listing}
  \inputminted{racket}{listings/long-queue.rkt}
  \caption{\texttt{long-queue.rkt} module}
  \label{lst:long-queue.rkt}
\end{listing}

\begin{listing}
  \inputminted{racket}{listings/short-queue.rkt}
  \caption{\texttt{short-queue.rkt} module}
  \label{lst:short-queue.rkt}
\end{listing}


\section{Compilation}

Compiling this program involves expanding the \racket{with-queue} macro, which will turn the program into Listing~\ref{main-expanded.rkt}.

\begin{listing}
  \inputminted{racket}{listings/main-expanded.rkt}
  \caption{\texttt{main.rkt} module after macro expansion}
  \label{lst:main-expanded.rkt}
\end{listing}

The expanded program contains references to \racket{long-queue} operations because the length of the initial queue was over 5.
While the expanded program is not the same as a fully compiled program, it can stand in for one with regard to optimization and demodularization.

With the expanded program, it is possible to see how cross-module optimizations would be difficult.
The program has calls to functions in the \racket{long-queue} module, so the optimizer does not have access to the definitions of the functions while finishing the compilation of the main module.

\section{Demodularization}

Demodularization of the program proceeds by following the \racket{require} graph and including phase 0 code. 
The \racket{main} module requires the \racket{queue} module, but the \racket{queue} module only has phase 1 code, so it is not included in the final module.
The \racket{queue} module requires the \racket{long-queue} and \racket{short-queue} modules, which have various phase 0 definitions that get included in the order they appear in their original modules.
Listing~\ref{lst:main-demodularized.rkt} shows what the demodularized example program would look like.

\begin{listing}
  \inputminted{racket}{listings/main-demodularized.rkt}
  \caption{\texttt{main.rkt} module after demodularization}
  \label{lst:main-demodularized.rkt}
\end{listing}

% TODO: make a point that extra queue implementation is there
The demodularized program should have the exact same behavior at runtime as the modular program.
At this point, the only thing that changes is that the Racket runtime does not need to follow the \racket{require} graph because allphase 0 definitions and expressions are in a single module.

\section{Optimization}

After a program has been demodularized, the optimizer now has all the information it needs to do optimizations on the whole program.
Listing~\ref{lst:main-optimized.rkt} shows what the example program would look like after optimization.

\begin{listing}
  \inputminted{racket}{listings/main-optimized.rkt}
  \caption{\texttt{main.rkt} module after optimization}
  \label{lst:main-optimized.rkt}
\end{listing}

% TODO: point out that the dead code is eliminated
The optimizer is able to inline the various functions because it has access to all of the definitions within the module.
Therefore, demodularization enables optimization by providing the optimizer access to the whole program in one place.
