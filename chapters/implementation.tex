\chapter{Implementation}
\label{chap:implementation}

The implementation of demodularization for the Racket module system integrates with the existing compilation process. 
The goal of the implementation was to write the actual algorithm in Racket, but to use as much of the existing infrastructure as possible. 
We did not want to duplicate work done by the compiler, so the demodularization algorithm operates on Racket bytecode. 
We also did not want to duplicate work done by the optimizer, so we pass the demodularized program through the existing optimizer.
All of the pieces of the demodularization process are contained within a command line tool. 

\section{Racket Compilation Process}

Figure~\ref{fig:compilation} shows the compilation process that a Racket module undergoes and the various intermediate forms the code takes.
\begin{figure}
  \begin{tikzpicture}[rep/.style={draw, text width=6cm, align=center}]
  \node (racket) [rep] {Racket code (.rkt)};
  \node (expanded) [rep, below=of racket] {Fully expanded code};
  \node (IR) [rep, below=of expanded] {Intermediate Representation};
  \node (OIR) [rep, below=of IR] {Optimized IR};
  \node (zo) [rep, below=of OIR] {Racket bytecode (.zo)};

  \draw[->] (racket)--(expanded) node [midway, right] {macro expander};
  \draw[->] (expanded)--(IR) node [midway, right] {compiler};
  \draw[->] (IR)--(OIR) node [midway, right] {optimizer};
  \draw[->] (OIR)--(zo) node [midway, right] {compiler};

\end{tikzpicture}

  \label{fig:compilation}
  \caption{Racket compilation process}
\end{figure}
In principle it would be possible to write the demodularization algorithm for modules at any point in this process, but we chose to write it at the bytecode level because it is a well defined file type and phases are clearly deliniated.

The compilation process for Racket is all written in C.
The first step of compiling a Racket module is to expand all of the macros it uses into what is known as the Racket kernel language or fully expanded code.
The next step is compilation to an Intermediate Representation (IR) made up of C data structures.
The optimizer works on this IR to produce an optmized version of it.
Then, the compiler finishes turning the IR into bytecode.

\section{Racket bytecode}
Racket bytecode is different from most other forms of bytecode in that it mostly maintains the structure of the abstract syntax tree from the original program.
The main aspect that changes between fully-expanded Racket programs and Racket bytecode is the use of stack positions instead of variables and the addition of stack-manipulating instructions.
A full explanation of Racket bytecode and what it means can be found in \cite{bytecode}.

% DONE: Don't start a paragraph with Also
A table is created in the bytecode for the top level bindings in a module, which is known as the prefix.
All top level bindings have an entry in the prefix, including bindings that come from other modules, and any use of a top level binding in a program is replaced with a numeric reference to the entry.
Listing~\ref{lst:kernel.rkt} shows a simple program written in the most basic language that Racket supports: the \racket{kernel} language.
\begin{listing}
  \inputminted{racket}{listings/kernel.rkt}
  \caption{Example program written in \racket{kernel} language}
  \label{lst:kernel.rkt}
\end{listing}
When this code is compiled, it turns into the bytecode in Listing~\ref{lst:kernel-bytecode.rkt}. 
\begin{listing}
  \inputminted{racket}{listings/kernel-bytecode.rkt}
  \caption{Bytecode representation of program from Listing~\ref{lst:kernel.rkt}}
  \label{lst:kernel-bytecode.rkt}
\end{listing}

% DONE: comment on primval
In the bytecode, all references to the variable \racket{y} have been replaced with references to the prefix in the form of \racket{(toplevel 0)}. 
All references to \racket{displayln} have also been replaced with \racket{(toplevel 1)} references, which in turn references element \racket{7} of the prefix of the \racket{racket/private/misc} module.
The references to \racket{x} are now all different because the stack changes between each usage of \racket{x}.
The references to \racket{random} and \racket{+} are replaced with references to primitive implementations in the Racket runtime. 

\section{Demodularization}

The implementation of demodularization for the Racket module system is written in Racket and consumes and produces Racket bytecode.
Figure~\ref{fig:demod} shows the demodularization process and how it interacts with existing racket components.
The library for reading and writing Racket bytecode in Racket was mainly used for debugging purposes and was incomplete, so the first task was to fully implement the Racket bytecode library.
\begin{figure}
  \begin{tikzpicture}[rep/.style={draw, text width=6cm, align=center}]
  \node (modules) [rep] {Bytecode modules (.zo)};
  \node (racket) [rep, below=of modules] {Modules in Racket};
  \node (demod) [rep, below=of racket] {Single module in Racket};
  \node (demod-zo) [rep, below=of demod] {Single bytecode module (.zo)};
  \node (IR) [rep, below=of demod-zo] {Single module in IR};
  \node (opt-zo) [rep, below=of IR] {Optimized bytecode module (.zo)};

  \draw[->] (modules)--(racket) node [midway, right] {bytecode reader};
  \draw[->] (racket)--(demod) node [midway, right] {demodularizer};
  \draw[->] (demod)--(demod-zo) node [midway, right] {bytecode writer};
  \draw[->] (demod-zo)--(IR) node [midway, right] {decompiler};
  \draw[->] (IR)--(opt-zo) node [midway, right] {optimizer, compiler};
\end{tikzpicture}

  \label{fig:demod}
  \caption{Demodularization process}
\end{figure}

The actual algorithm for demodularization is similar to the model in the previous chapter in that it traces requires and includes all phase 0 code in the final module, but it also has to deal with the module prefix and references to the prefix.
When the algorithm includes a required module, it also includes that module's prefix appended to the end of the main module's prefix. 
Then, it must adjust all references in that module's code to point at the new combined prefix.
Also, the algorithm tracks cross-module references and rewrites them to point to the new prefix as well.
In the example in Listing~\ref{lst:kernel-bytecode.rkt}, when the algorithm includes the module \racket{racket/private/misc} it must rewrite the reference to \racket{(toplevel 1)} to whatever the new position for \racket{displayln} is in the combined prefix.
% TODO: show a demod example

\section{Optimization}

For demodularization to be useful, the program needs to be optimized after demodularizing it.
% TODO: relate back to thesis rather than say we wanted
To optmize the demodularized bytecode, we wanted the existing Racket optimizer built into the Racket compiler so that we get all existing optimizations ``for free'' and any new optimizations in the future will work on both regularly compiled and demodularized programs.
The existing optimizer works on an intermediate form between fully-expanded Racket code and Racket bytecode.
This intermediate form only exists as C data structures in the implementation of Racket.
Therefore, to use the existing optimizer, Racket bytecode needed to be decompiled into this intermediate form. 

\section{Decompilation}

The decompiler was written in C, so that it could interoperate with the optimizer.
There are three main differences between Racket bytecode and the intermediate form: stack positions, cyclic closures, and reference arguments.
In Racket bytecode and in the intermediate form, all references to bindings are numeric, but in bytecode the references are stack positions and in the intermediate form the references are lexical.
For example, the bytecode in Listing~\ref{lst:kernel-bytecode.rkt} has three references to \racket{x}, in the form of \racket{(install-value 0 5)}, \racket{(local 1)}, and \racket{(local 3)}. 
% DONE: explain why they should all be 0
In the intermediate form, all of these references to \racket{x} should be \racket{0} because lexically they all refer to the same variable that is the closest binding.

Racket bytecode allows for cyclic closures, or closures which contain a reference to themselves.
Nothing like this exists in the intermediate form, so to decompile cyclic closures, the decompiler creates a new top level definition for the closure and replaces references to the closure (including the self-references) with references to the top level definition.
% TODO: maybe an example of this as well

Finally, Racket bytecode allows reference arguments (like C++ reference arguments) in functions, but the intermediate form doesn't allow them.
The decompiler turns reference arguments into \racket{case-lamdba} closures over the reference arguments with one case for getting the argument value and one case for setting the argument value. 

% TODO: how to know if it works (verification) paragraph
The decompilation algorithm is not the identity function when composed with compilation because of the transformations performed on cyclic closures and reference arguments.
Sometimes the optimizer will remove the \racket{case-lambda} arguments and turn them back into reference arguments, but this is not guaranteed. 
Racket has robust bytecode verification built in to the module loading system, so errors that were introduced by the decompiler were mostly caught by bytecode verification.

\section{Limitations}

Racket provides features that treat modules as first-class objects during runtime. 
For example, programs can load and evaluate modules at runtime through \racket{dynamic-require}. 
Because the demodularizer cannot know ahead of time what modules might be loaded at runtime, it disallows programs that use \racket{dynamic-require}.
If the modules loaded through \racket{dynamic-require} were completely separate (meaning they do not share any required modules) from the main program, it would be possible to demodularize the program, but in practice most modules required at runtime will share with the main program.

The restriction of not allowing \racket{dynamic-require} means that programs that need to load modules on the fly, such as REPLs, sandboxed evaluation environments, and scripting environments cannot be demodularized.
This is okay because such programs are fundamentally incomplete programs whose full meaning isn't known until runtime.

\section{Usage}
The implementation of demodularization is included in the Racket distribution as part of the build tool \racket{raco}. 
Users can pass in a module they would like to demodularize and optmize to the tool and it will compile all the necessary modules, run the demodularization algorithm on them, decompile the resulting module, and run optimizations on it to produce the final module.
For example, to demodularize a program whose main module was \texttt{while-test.rkt}, a user would type the following command: 

\begin{verbatim}
raco demod -O while-test.rkt
\end{verbatim}

