\chapter{Introduction}
\label{chap:introduction}

Programmers should not have to sacrifice the software engineering goals of modular design and good abstractions for performance. 
Instead, their tools should make running a well-designed program as efficient as possible. 

Many programming languages provide features for creating modular programs. 
This may be as simple as separating code into different files or as complex as specifying separate interfaces and implementations for modules.
When code is separated into modules, it is possible to compile each module separately provided enough information is known about calls to other modules.
Separate compilation allows programmers to work on one part of their program without needing to recompile the whole program, but this convenience comes at a cost.
The compiler does not have much information about functionality from other modules as it compiles a single module, making optimizations more difficult.

A separately compiled program is turned into an executable through linking all of the modules together, which can happen either statically or dynamically.
Static linking creates a full executable program by combining all of the modules of the program before running the program.
Dynamic linking creates references to other modules in the program and loads them as needed when the program runs.
Linking is just one step in the process of taking a program from source code to running code, and programming languages often have more steps that happen in that transformation process.
This research looks at the impact macro systems have on optimizing separately compiled programs.

Some programming languages provide macro systems, which run before compiling, that enable programmers to add features to the programming language through manipulation of the syntax of the program.
Simple macro systems allow textual replacement, like the preprocessor in C/C++, while more advanced systems provide access to the syntax as objects with rich lexical information and a full programming language to manipulate the syntax.
Some of these more advanced macro systems even allow programmers to write macros in the same programming language they use to write their programs. 

These advanced macro systems give programmers the ability to extend a programming language in arbitrary ways, even to the point of being able to create domain specific languages as collections of macros.
When creating such programs, the details of individual macros are often hidden in modules, and, much like a separately compiled program, these modules need to be linked into an executable. 
Unlike a separately compiled program though, macros generate code, so they need to run before the program can be compiled.

Macros written in the same language as the program need to use the same compiler that the language uses, but the compilation needs to happen before normal program compilation.
Macros can also use other macros in their implementation, which means that those other macros must be ready to run (compiled) before their use.
This leads to an interleaving of compiling and running code that goes back and forth to ultimately produce the final program.

Separate compilation is more difficult in the presence of an advanced macro system.
Macros are defined side-by-side with program definitions, which means that macro definitions can be spread across modules and can also use functionality provided in separate modules.
A compiler for such a language must ensure that modular programs have the same meaning independent of the order in which the modules are compiled.

A solution to the problems of separate compilation with advanced macros is described by Flatt and implemented for the Racket programming language \cite{Flatt}.
He explains a method of compiling modules in phases, where each phase corresponds to which part of the program is running and which part of the program is compiling.
In this case, running means executing code, which includes executing macros, because macros are just code as well.
Compiling means translating code from source code to executable code, perhaps with macros doing part of the translation.
At phase 0, the main program is running and nothing is compiling.
At phase 1, the main program is compiling and macros that it uses are running.
At phase 2, the phase 1 macros are compiling and any macros used in the compilation of phase 1 macros are running, and so on.
The compiler starts compiling at the highest phase of the program (which can be upwards of phase 70 in normal Racket programs) and compiles each phase downward until it produces phase 0 code. 
By separating compilation into phases, it is possible to have both a module system and a macro system coexist.

Separate compilation makes compiler optimization more difficult and less effective.
Good abstractions, provided by splitting a program into modules, are meant to obscure internal implementations so that it is easier for programmers to reason about their programs, but this obscurity also limits information available for optimizations. 
A single module compiled alone is difficult to optimize because the compiler has little to no information about values that come from other modules when compiling the module.
Existing optimizations have even less information when modules can extend the compiler, as modules can in languages with advanced macro systems. 

Language implementations have solved the problem of optimizing separately compiled programs in various ways.
Some implementations of languages avoid the problems associated with separate compilation by not allowing it. 
The whole-program optimizing compilers Stalin \cite{stalin} for Scheme and MLton \cite{mlton} for ML take this approach.
Other implementations do optimizations while statically linking the program.
There are options in gcc \cite{gcc} and clang \cite{clang} to do this type of optimization.
This type of optimization is usually done on machine code, so it can be too low level to do certain optimizations.
Some implementations perform cross-module inlining by selectively choosing functions to inline between modules.
Inlining must be heuristic-based, and good heuristics are hard to develop. 
Just-In-Time (JIT) compilers attack the problem in a different way by deferring optimizations until the program is running, where it has more information on the actual use of the program.

Our solution for optimizing modular programs in the presence of an advanced macro system, called demodularization, is to transform a modular program into a non-modular program by combining all phase 0 (runtime) code and data in the program into a single module.
Conceptually, this is similar to what static linkers do in programming languages with separate compilation, but more complicated because finding all phase 0 code from the set of modules that comprise a program is not trivial.
Modules can include other modules at different phase offsets, creating a directed acyclic graph of module relationships.
In finding and combining all of the phase 0 code, the demodularization algorithm removes code for all other phases because they were only needed to compile the program, not run it.

After combining, the single module is then re-run through the existing optimizer, but this time the optimizer has information about the whole program.
This part is similar to doing link-time optimization for programming languages with separate compilation, but is at a higher level than most link-time optimizers.
Demodularization is meant to be done when producing the final production version of a program, so that during development, a programmer can still take advantage of separate compilation.

\vspace{1em}
\noindent
\textbf{Thesis} \textit{Demodularizing programs in a language with a module system and an expressive macro system is feasible and yields more optimization opportunity for the compiler, resulting in more efficient code when compared to phased compilation and cross-module inlining.}
\vspace{1em}

The approach we take to validating the thesis is to first describe the Racket module and macro systems in Chapter~\ref{chap:module-system} as an example of a language with a module system and an expressive macro system that is open source.
Then, in Chapter~\ref{chap:intuition}, we explain demodularization at a high level with a detailed example to show how the algorithm works on a Racket program.
Next, we use an operational semantics model of the demodularization process on a simple language to show that demodularization preserves program behavior (Chapter~\ref{chap:model}).
The operational semantics model removes the unnecessary details of the full implementation so the demodularization process is easier to understand and verify. 
We then describe an actual implementation for Racket (Chapter~\ref{chap:implementation}).
The implementation presents interesting difficulties in using the Racket bytecode format and integrating with the existing compiler framework written in C.  
Following that are experimental results of demodularizing and optimizing Racket programs and comparing them to phased compilation and cross-module inlining (Chapter~\ref{chap:evaluation}). 
The experimental results show that demodularization improves performance, especially when a program is highly modular.
Finally, we discuss related work on whole program optimization (Chapter~\ref{chap:related-work}) and conclude with a summary and discussion of future work (Chapter~\ref{chap:conclusion}). 
