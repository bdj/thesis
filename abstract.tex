\documentabstract{%
Programmers want to write modular programs to increase maintainability and create abstractions, but modularity hampers optimizations, especially when modules are compiled separately or written in different languages. 
In languages with syntactic extension capabilities, each module in a program can be written in a separate language, and the module system must ensure that the modules interoperate correctly. 
In Racket, the module system ensures this by separating module code into phases for runtime and compile-time and allowing phased imports and exports inside modules. 
We present an algorithm, called demodularization, that combines all executable code from a phased modular program into a single module that can then be optimized as a whole program. 
The demodularized programs have the same behavior as their modular counterparts but are easier to optimize. 
We show that programs maintain their meaning through an operational semantics of the demodularization process and verify that performance increases by running the algorithm and existing optimizations on Racket programs.
}
